%%%%%%%%%%%%%%%%%%%%%%%%%%%%%%%%%%%%%%%%%%%%%%%%%%%%%%%%%%%%%%%%%
% Contents: TeX and LaTeX and AMS symbols for Maths
% $Id: lssym.tex,v 1.2 2003/03/19 20:57:46 oetiker Exp $
%%%%%%%%%%%%%%%%%%%%%%%%%%%%%%%%%%%%%%%%%%%%%%%%%%%%%%%%%%%%%%%%%
% 中文~4.20~翻译:liwenjun@bbs.ctex  email:sydlee@gmail.com
%%%%%%%%%%%%%%%%%%%%%%%%%%%%%%%%%%%%%%%%%%%%%%%%%%%%%%%%%%%%%%%%%

%\section{List of Mathematical Symbols}  \label{symbols}
\section{数学符号表}  \label{symbols}
%The following tables demonstrate all the symbols normally accessible from \emph{math mode}.
以下表格列出了\textbf{数学模式}中的所有常用符号。
%
% Conditional Text in case the AMS Fonts are installed
%
%To use the symbols listed in
%Tables~\ref{AMSD}--\ref{AMSNBR},\footnote{These tables were derived
%  from \texttt{symbols.tex} by David~Carlisle and subsequently changed
%extensively as suggested by Josef~Tkadlec.} the package
%\pai{amssymb} must be loaded in the preamble of the document and the
%AMS math fonts must be installed on the system. If the AMS package and
%fonts are not installed on your system, have a look at\\
%\CTANref|macros/latex/required/amslatex|. An even more comprehensive list of
%symbols can be found at  \CTANref|info/symbols/comprehensive|.

要使用表 \ref{AMSD}--\ref{AMSNBR}\footnote{这些表格源自于 David Carlisle 的
\texttt{symbols.tex}, 而后在 Josef Tkadlec 的建议下作了较大的改动。}。
必须在导言区先载入 \pai{amssymb} 宏包而且
系统中安装了 AMS 数学字体。如果系统中没有安装 AMS 宏包和字体,请查阅 \CTANref|macros/latex/required/amslatex|。
更全面的列表可于 \CTANref|info/symbols/comprehensive| 处找到。
\begin{table}[!h]
\caption{数学模式重音符号。}  \label{mathacc}
\begin{symbols}{*3{cl}}
\W{\hat}{a}     & \W{\check}{a} & \W{\tilde}{a} \\
\W{\grave}{a} & \W{\dot}{a} & \W{\ddot}{a}     \\
\W{\bar}{a} &\W{\vec}{a} &\W{\widehat}{A}  \\
\W{\acute}{a}  & \W{\breve}{a} &\W{\widetilde}{A}
\end{symbols}
\end{table}

\begin{table}[!h]
\caption{希腊字母。}
\begin{symbols}{*4{cl}}
 \X{\alpha}     & \X{\theta}     & \X{o}          & \X{\upsilon}  \\
 \X{\beta}      & \X{\vartheta}  & \X{\pi}        & \X{\phi}      \\
 \X{\gamma}     & \X{\iota}      & \X{\varpi}     & \X{\varphi}   \\
 \X{\delta}     & \X{\kappa}     & \X{\rho}       & \X{\chi}      \\
 \X{\epsilon}   & \X{\lambda}    & \X{\varrho}    & \X{\psi}      \\
 \X{\varepsilon}& \X{\mu}        & \X{\sigma}     & \X{\omega}    \\
 \X{\zeta}      & \X{\nu}        & \X{\varsigma}  &               \\
 \X{\eta}       & \X{\xi}        & \X{\tau} & \\
 \X{\Gamma}     & \X{\Lambda}    & \X{\Sigma}     & \X{\Psi}      \\
 \X{\Delta}     & \X{\Xi}        & \X{\Upsilon}   & \X{\Omega}    \\
 \X{\Theta}     & \X{\Pi}        & \X{\Phi}
\end{symbols}
\end{table}



\begin{table}[!tbp]
\caption{二元关系。}
\bigskip
%You can negate the following symbols by prefixing them with a \ci{not} command.
你可以在下列符号的相应命令前加上~\ci{not}~命令,而得到其否定形式。
\begin{symbols}{*3{cl}}
 \X{<}           & \X{>}           & \X{=}          \\
 \X{\leq}or \verb|\le|   & \X{\geq}or \verb|\ge|   & \X{\equiv}     \\
 \X{\ll}         & \X{\gg}         & \X{\doteq}     \\
 \X{\prec}       & \X{\succ}       & \X{\sim}       \\
 \X{\preceq}     & \X{\succeq}     & \X{\simeq}     \\
 \X{\subset}     & \X{\supset}     & \X{\approx}    \\
 \X{\subseteq}   & \X{\supseteq}   & \X{\cong}      \\
 \X{\sqsubset}$^a$ & \X{\sqsupset}$^a$ & \X{\Join}$^a$    \\
 \X{\sqsubseteq} & \X{\sqsupseteq} & \X{\bowtie}    \\
 \X{\in}         & \X{\ni}, \verb|\owns|  & \X{\propto}    \\
 \X{\vdash}      & \X{\dashv}      & \X{\models}    \\
 \X{\mid}        & \X{\parallel}   & \X{\perp}      \\
 \X{\smile}      & \X{\frown}      & \X{\asymp}     \\
 \X{:}           & \X{\notin}      & \X{\neq}or \verb|\ne|
\end{symbols}
%\centerline{\footnotesize $^a$Use the \textsf{latexsym} package to access this symbol}
\centerline{\footnotesize
$^a$~使用~\textsf{latexsym}~宏包才能得到这个符号}
\end{table}

\begin{table}[!tbp]
\caption{二元运算符。}
\begin{symbols}{*3{cl}}
 \X{+}              & \X{-}              & &                 \\
 \X{\pm}            & \X{\mp}            & \X{\triangleleft} \\
 \X{\cdot}          & \X{\div}           & \X{\triangleright}\\
 \X{\times}         & \X{\setminus}      & \X{\star}         \\
 \X{\cup}           & \X{\cap}           & \X{\ast}          \\
 \X{\sqcup}         & \X{\sqcap}         & \X{\circ}         \\
 \X{\vee}, \verb|\lor|     & \X{\wedge}, \verb|\land|  & \X{\bullet}       \\
 \X{\oplus}         & \X{\ominus}        & \X{\diamond}      \\
 \X{\odot}          & \X{\oslash}        & \X{\uplus}        \\
 \X{\otimes}        & \X{\bigcirc}       & \X{\amalg}        \\
 \X{\bigtriangleup} &\X{\bigtriangledown}& \X{\dagger}       \\
 \X{\lhd}$^a$         & \X{\rhd}$^a$         & \X{\ddagger}      \\
 \X{\unlhd}$^a$       & \X{\unrhd}$^a$       & \X{\wr}
\end{symbols}

\end{table}

\begin{table}[!tbp]
\caption{“大”运算符。}
\begin{symbols}{*4{cl}}
 \X{\sum}      & \X{\bigcup}   & \X{\bigvee}  \\
 \X{\prod}     & \X{\bigcap}   & \X{\bigwedge} \\
 \X{\coprod}   & \X{\bigsqcup} & \X{\biguplus} \\
 \X{\int}      & \X{\oint}     & \X{\bigodot} \\
 \X{\bigoplus} & &\X{\bigotimes} & \\
\end{symbols}

\end{table}


\begin{table}[!tbp]
\caption{箭头。}
\begin{symbols}{*2{cl}}
 \X{\leftarrow}or \verb|\gets|& \X{\longleftarrow} \\
 \X{\rightarrow}or \verb|\to|& \X{\longrightarrow} \\
 \X{\leftrightarrow}    & \X{\longleftrightarrow} \\
 \X{\Leftarrow}         & \X{\Longleftarrow}     \\
 \X{\Rightarrow}        & \X{\Longrightarrow}    \\
 \X{\Leftrightarrow}    & \X{\Longleftrightarrow}\\
 \X{\mapsto}            & \X{\longmapsto}        \\
 \X{\hookleftarrow}     & \X{\hookrightarrow}    \\
 \X{\leftharpoonup}     & \X{\rightharpoonup}    \\
 \X{\leftharpoondown}   & \X{\rightharpoondown}  \\
 \X{\rightleftharpoons} & \X{\iff}(bigger spaces) \\
 \X{\uparrow}   & \X{\downarrow} \\
 \X{\updownarrow} & \X{\Uparrow} \\
 \X{\Downarrow} &  \X{\Updownarrow} \\
 \X{\nearrow} &  \X{\searrow} \\
  \X{\swarrow} & \X{\nwarrow} \\
 \X{\leadsto}$^a$
\end{symbols}
\centerline{\footnotesize $^a$使用~\textsf{latexsym}~宏包才能得到这个符号}
\end{table}

\begin{table}[!tbp]
\caption{定界符。}\label{tab:delimiters}
\begin{symbols}{*3{cl}}
 \X{(}            & \X{)}            & \X{\uparrow} \\
 \X{[}or \verb|\lbrack|   & \X{]}or \verb|\rbrack|  & \X{\downarrow}   \\
 \X{\{}or \verb|\lbrace|  & \X{\}}or \verb|\rbrace|  & \X{\updownarrow} \\
 \X{\langle}      & \X{\rangle}  & \X{|}or \verb|\vert| \\
 \X{\lfloor}      & \X{\rfloor}      & \X{\lceil}       \\
 \X{/}            & \X{\backslash}   & \X{\Updownarrow}\\
 \X{\Uparrow}     &  \X{\Downarrow}  & \X{\|}or \verb|\Vert| \\
  \X{\rceil}
\end{symbols}
\end{table}

\begin{table}[!tbp]
\caption{大定界符。}
\begin{symbols}{*3{cl}}
 \Y{\lgroup}      & \Y{\rgroup}      & \Y{\lmoustache}  \\
 \Y{\arrowvert}   & \Y{\Arrowvert}   & \Y{\bracevert} \\
 \Y{\rmoustache} \\
\end{symbols}
\end{table}


\begin{table}[!tbp]
\caption{其他符号。}
\begin{symbols}{*4{cl}}
 \X{\dots}       & \X{\cdots}      & \X{\vdots}      & \X{\ddots}     \\
 \X{\hbar}       & \X{\imath}      & \X{\jmath}      & \X{\ell}       \\
 \X{\Re}         & \X{\Im}         & \X{\aleph}      & \X{\wp}        \\
 \X{\forall}     & \X{\exists}     & \X{\mho}$^a$      & \X{\partial}   \\
 \X{'}           & \X{\prime}      & \X{\emptyset}   & \X{\infty}     \\
 \X{\nabla}      & \X{\triangle}   & \X{\Box}$^a$     & \X{\Diamond}$^a$ \\
 \X{\bot}        & \X{\top}        & \X{\angle}      & \X{\surd}      \\
\X{\diamondsuit} & \X{\heartsuit}  & \X{\clubsuit}   & \X{\spadesuit} \\
 \X{\neg}or \verb|\lnot| & \X{\flat}       & \X{\natural}    & \X{\sharp}

\end{symbols}
\centerline{\footnotesize $^a$使用~\textsf{latexsym}~宏包才能得到这个符号}
\end{table}


\begin{table}[!tbp]
\caption{非数学符号。}
\bigskip
%These symbols can also be used in text mode.
也可以在文本模式中使用这些符号。
\begin{symbols}{*4{cl}}
 \SC{\dag}  &  \SC{\S}  &  \SC{\copyright} &  \SC{\textregistered}  \\
 \SC{\ddag} &  \SC{\P}  &  \SC{\pounds}    &  \SC{\%}               \\
\end{symbols}
\end{table}

%
%
% If the AMS Stuff is not available, we drop out right here :-)
%

\begin{table}[!tbp]
\caption{AMS~定界符。}\label{AMSD}
\bigskip
\begin{symbols}{*4{cl}}
\X{\ulcorner}&\X{\urcorner}&\X{\llcorner}&\X{\lrcorner}\\
\X{\lvert}&\X{\rvert}&\X{\lVert}&\X{\rVert}
\end{symbols}
\end{table}

\begin{table}[!tbp]
\caption{AMS~希腊和希伯来字母。}
\begin{symbols}{*5{cl}}
\X{\digamma}     &\X{\varkappa} & \X{\beth} &\X{\gimel} & \X{\daleth}
\end{symbols}
\end{table}

\begin{table}[!tbp]
\caption{AMS~二元关系。}
\begin{symbols}{*3{cl}}
 \X{\lessdot}           & \X{\gtrdot}            & \X{\doteqdot} \\
 \X{\leqslant}          & \X{\geqslant}          & \X{\risingdotseq}     \\
 \X{\eqslantless}       & \X{\eqslantgtr}        & \X{\fallingdotseq}    \\
 \X{\leqq}              & \X{\geqq}              & \X{\eqcirc}           \\
 \X{\lll}or \verb|\llless| & \X{\ggg}            & \X{\circeq}  \\
 \X{\lesssim}           & \X{\gtrsim}            & \X{\triangleq}        \\
 \X{\lessapprox}        & \X{\gtrapprox}         & \X{\bumpeq}           \\
 \X{\lessgtr}           & \X{\gtrless}           & \X{\Bumpeq}           \\
 \X{\lesseqgtr}         & \X{\gtreqless}         & \X{\thicksim}         \\
 \X{\lesseqqgtr}        & \X{\gtreqqless}        & \X{\thickapprox}      \\
 \X{\preccurlyeq}       & \X{\succcurlyeq}       & \X{\approxeq}         \\
 \X{\curlyeqprec}       & \X{\curlyeqsucc}       & \X{\backsim}          \\
 \X{\precsim}           & \X{\succsim}           & \X{\backsimeq}        \\
 \X{\precapprox}        & \X{\succapprox}        & \X{\vDash}            \\
 \X{\subseteqq}         & \X{\supseteqq}         & \X{\Vdash}            \\
 \X{\shortparallel}     & \X{\Supset}            & \X{\Vvdash}           \\
 \X{\blacktriangleleft} & \X{\sqsupset}          & \X{\backepsilon}      \\
 \X{\vartriangleright}  & \X{\because}           & \X{\varpropto}        \\
 \X{\blacktriangleright}& \X{\Subset}            & \X{\between}          \\
 \X{\trianglerighteq}   & \X{\smallfrown}        & \X{\pitchfork}        \\
 \X{\vartriangleleft}   & \X{\shortmid}      & \X{\smallsmile}  \\
 \X{\trianglelefteq}    & \X{\therefore}     & \X{\sqsubset}
\end{symbols}
\end{table}

\begin{table}[!tbp]
\caption{AMS~箭头。}
\begin{symbols}{*2{cl}}
 \X{\dashleftarrow}      & \X{\dashrightarrow}     \\
 \X{\leftleftarrows}     & \X{\rightrightarrows}   \\
 \X{\leftrightarrows}    & \X{\rightleftarrows}    \\
 \X{\Lleftarrow}         & \X{\Rrightarrow}        \\
 \X{\twoheadleftarrow}   & \X{\twoheadrightarrow}  \\
 \X{\leftarrowtail}      & \X{\rightarrowtail}     \\
 \X{\leftrightharpoons}  & \X{\rightleftharpoons}  \\
 \X{\Lsh}                & \X{\Rsh}                \\
 \X{\looparrowleft}      & \X{\looparrowright}     \\
 \X{\curvearrowleft}     & \X{\curvearrowright}    \\
 \X{\circlearrowleft}    & \X{\circlearrowright}   \\
 \X{\multimap}  &  \X{\upuparrows}  \\
 \X{\downdownarrows} & \X{\upharpoonleft} \\
 \X{\upharpoonright} & \X{\downharpoonright} \\
 \X{\rightsquigarrow} & \X{\leftrightsquigarrow} \\
\end{symbols}
\end{table}

\begin{table}[!tbp]
\caption{AMS~二元否定关系符和箭头。}\label{AMSNBR}
\begin{symbols}{*3{cl}}
 \X{\nless}           & \X{\ngtr}            & \X{\varsubsetneqq}  \\
 \X{\lneq}            & \X{\gneq}            & \X{\varsupsetneqq}  \\
 \X{\nleq}            & \X{\ngeq}            & \X{\nsubseteqq}     \\
 \X{\nleqslant}       & \X{\ngeqslant}       & \X{\nsupseteqq}     \\
 \X{\lneqq}           & \X{\gneqq}           & \X{\nmid}           \\
 \X{\lvertneqq}       & \X{\gvertneqq}       & \X{\nparallel}      \\
 \X{\nleqq}           & \X{\ngeqq}           & \X{\nshortmid}      \\
 \X{\lnsim}           & \X{\gnsim}           & \X{\nshortparallel} \\
 \X{\lnapprox}        & \X{\gnapprox}        & \X{\nsim}           \\
 \X{\nprec}           & \X{\nsucc}           & \X{\ncong}          \\
 \X{\npreceq}         & \X{\nsucceq}         & \X{\nvdash}         \\
 \X{\precneqq}        & \X{\succneqq}        & \X{\nvDash}         \\
 \X{\precnsim}        & \X{\succnsim}        & \X{\nVdash}         \\
 \X{\precnapprox}     & \X{\succnapprox}     & \X{\nVDash}         \\
 \X{\subsetneq}       & \X{\supsetneq}       & \X{\ntriangleleft}  \\
 \X{\varsubsetneq}    & \X{\varsupsetneq}    & \X{\ntriangleright} \\
 \X{\nsubseteq}       & \X{\nsupseteq}       & \X{\ntrianglelefteq}\\
 \X{\subsetneqq}      & \X{\supsetneqq}      &\X{\ntrianglerighteq}\\[0.5ex]
 \X{\nleftarrow}      & \X{\nrightarrow}     & \X{\nleftrightarrow}\\
 \X{\nLeftarrow}      & \X{\nRightarrow}     & \X{\nLeftrightarrow}

\end{symbols}
\end{table}

\begin{table}[!tbp]
\caption{AMS~二元运算符。}
\begin{symbols}{*3{cl}}
 \X{\dotplus}        & \X{\centerdot}      &       \\
 \X{\ltimes}         & \X{\rtimes}         & \X{\divideontimes} \\
 \X{\doublecup}      & \X{\doublecap}      & \X{\smallsetminus} \\
 \X{\veebar}         & \X{\barwedge}       & \X{\doublebarwedge}\\
 \X{\boxplus}        & \X{\boxminus}       & \X{\circleddash}   \\
 \X{\boxtimes}       & \X{\boxdot}         & \X{\circledcirc}   \\
 \X{\intercal}       & \X{\circledast}     & \X{\rightthreetimes} \\
 \X{\curlyvee}       & \X{\curlywedge}     & \X{\leftthreetimes}
\end{symbols}
\end{table}

\begin{table}[!tbp]
\caption{AMS~其他符号。}
\begin{symbols}{*3{cl}}
 \X{\hbar}             & \X{\hslash}           & \X{\Bbbk}            \\
 \X{\square}           & \X{\blacksquare}      & \X{\circledS}        \\
 \X{\vartriangle}      & \X{\blacktriangle}    & \X{\complement}      \\
 \X{\triangledown}     &\X{\blacktriangledown} & \X{\Game}            \\
 \X{\lozenge}          & \X{\blacklozenge}     & \X{\bigstar}         \\
 \X{\angle}            & \X{\measuredangle}    & \\
 \X{\diagup}           & \X{\diagdown}         & \X{\backprime}       \\
 \X{\nexists}          & \X{\Finv}             & \X{\varnothing}      \\
 \X{\eth}              & \X{\sphericalangle}   & \X{\mho}
\end{symbols}
\end{table}



\begin{table}[!tbp]
\caption{数学字母。}
\begin{symbols}{@{}*3l@{}}
实例& 命令 &所需宏包\\
\hline
\rule{0pt}{1.05em}$\mathrm{ABCDE abcde 1234}$
        & \verb|\mathrm{ABCDE abcde 1234}|
        &       \\
$\mathit{ABCDE abcde 1234}$
        & \verb|\mathit{ABCDE abcde 1234}|
        &       \\
$\mathnormal{ABCDE abcde 1234}$
        & \verb|\mathnormal{ABCDE abcde 1234}|
        &  \\
$\mathcal{ABCDE abcde 1234}$
        & \verb|\mathcal{ABCDE abcde 1234}|
        &  \\
$\mathscr{ABCDE abcde 1234}$
        &\verb|\mathscr{ABCDE abcde 1234}|
        &\pai{mathrsfs}\\
$\mathfrak{ABCDE abcde 1234}$
        & \verb|\mathfrak{ABCDE abcde 1234}|
        &\pai{amsfonts}  or \textsf{amssymb}  \\
$\mathbb{ABCDE abcde 1234}$
        & \verb|\mathbb{ABCDE abcde 1234}|
        &\pai{amsfonts}  or \textsf{amssymb} \\
\end{symbols}
\end{table}


\endinput

%

% Local Variables:
% TeX-master: "lshort2e"
% mode: latex
% mode: flyspell
% End:
